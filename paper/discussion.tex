

\section{Discussion and Conclusions\label{sec:discussion}}

As promised in \S\ref{sec:intro}, our awesome discussion is here.


{\it Some notes for later:}

The Blazhko periods of RRc stars exhibit a strongly bimodal distribution but RRab stars do not
\citep[see][]{2007MNRAS.377.1263S}. Do we see any bimodality? 

Double-mode (d type) RR Lyrae are defined by $P_1/P_0 \sim 0.744$.
Eventually: can we rerun the Lomb-Scargle analysis and extract
the PS ratio for periods in the range $0.735 \le P_1/P_0 \le 0.755$?
For an example, see Section 4 in \cite{2007MNRAS.377.1263S}. 


{\it Connect the end of discussion to:}

The Legacy Survey of Space and Time (LSST; \citealt{2019ApJ...873..111I}) will be an excellent survey for studying Blazhko effect
\citep{2022ApJS..258....4H} because it will have both a long temporal
baseline (10 years) and a large number of observations per object
(nominally 825; LSST Science Requirements Document\footnote{Available as ls.st/srd}).




