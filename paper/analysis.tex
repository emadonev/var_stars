

\section{Data Description and Analysis Methodology\label{sec:analysis}}

We describe here how the starting sample of RR Lyrae stars was selected, the two datesets used in this work,
and the method used to search for the Blazhko effect.

Ema can start the first draft, with information from emails and notebooks, and then we'll go from there...


\subsection{LINEAR Dataset}

The properties of asteroid LINEAR survey and its photometric re-calibration based on SDSS data are discussed in \cite{2011AJ....142..190S}.
Briefly, the LINEAR survey covered about 10,000 deg$^2$ of the northern sky in white light (no filters were used, see Figure 1 in \citealt{2011AJ....142..190S}),
with photometric errors ranging from $\sim$0.03 mag at an equivalent SDSS magnitude of $r=15$ to 0.20 mag at $r\sim18$. Light curves used
in this work include on average 250 data points, collected better December 2002 and March 2008.
 
A sample of $\sim$7,000 periodic variable stars with $r<17$ discovered in LINEAR data was robustly classified by \cite{2013AJ....146..101P}, including
about $\sim$4,000 field RR Lyrae stars detected to distances of about 30 kpc \citep{2013AJ....146...21S}. 

We focus here on ab type (fundamental mode) RR Lyrae because their larger number and amplitudes result in statistically
more reliable samples than those available for c type (first overtone) stars \cite{2016CoKon.105...61K}.
We started with XXX ab-type RR Lyrae, according to classifications from \cite{2013AJ....146..101P}.

We accessed LINEAR light curve data using astroML...


\subsection{ZTF Dataset}

The Zwicky Transient Facility (ZTF) is an optical time-domain survey that uses the Palomar 48 inch Schmidt telescope
and a camera with 47 deg$^2$ field of view \citep{2019PASP..131a8002B}. The data analyzed here was obtained with
SDSS-like $g$, $r$, and $i$ band filters. Light curves for objects in common with the LINEAR RR Lyrae sample typically
have smaller random photometric errors than LINEAR light curves because ZTF data are deeper (compared to LINEAR,
ZTF data have about 2-3 magnitudes fainter  $5\sigma$ depth).

We accessed ZTF light curve data using {\it ztfquery}...




\subsection{Lomb-Scargle Periodogram Analysis}

Here we describe details about how we contructed the initial sample for LS periodogram analysis and all the
steps we did up to the first step of search for the Blazhko effect... 

For a good example of writing style on this topic, see sections 3.1 and 3.2 in
\begin{verbatim}
  https://iopscience.iop.org/article/10.3847/1538-3881/acb596/pdf
\end{verbatim}

\subsection{Searching for the Blazhko Effect}

The periodic shape change of light curves for Blazhko stars is equivalent to periodic phase and amplitude changes of the
harmonics which make up the light curve. The effect can be identified in the Lomb-Scargle power spectrum (periodogram)
by additional peaks (called sidebands) around the peak corresponding to the main pulsation, at frequencies $f_-$ and $f_+$,
with $f_- < f_0 < f_+$, where $f_0$ is the frequency of the main pulsation. The sideband peaks can be highly asymmetric
\cite{2003ApJ...598..597A}. We note that observed periodograms can sometimes be much more complex \cite{2007MNRAS.377.1263S}. 

The Blazhko period is defined as
\begin{equation}
  P_{BL} = (f_{-,+} - f_0)^{-1},
\end{equation}
where $f_{-,+}$ simply means that the Blazhko sideband frequency with higher amplitude is chosen. 

The Blazhko periods range from 3 days to 3,000 days, and Blazhko amplitudes range from 0.01 mag to about 0.3 mag \cite{2007MNRAS.377.1263S}.

% According to Collinge, Sumi \& Fabrycky (2006) (perhaps Alcock et al.?): 
%
% First pre-whiten: subtract a Fourier series of best-fitting harmonics of the main pulsation period from the data.
% If $T$ is the overall time baseline of the observations for the given light curve, then $\Delta f = P_{BL}^{-1}$ must
% be larger than $1/T$ and smaller than 0.3 d$^{-1}$ (empirically, Blazhko periods are larger than 3 days).
% --> did we do that?

Based on the LS periodogram plot for LINEAR ID=20668112 and ZTF data, at least for that star we are
sensitive to Blazhko periods in the range 30 days (limited by the plotted frequency window width) to
1000 days (limited by the ability to see the sidebands separated from the main peak). 


Here we describe first how we constructed the sample of 805 stars (for example, what exactly happens in
{\it periodogram\_blazhko}), and then how exactly was this sample cut down to 192 stars with visual inspection... 