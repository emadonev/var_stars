
\section{Introduction\label{sec:intro}}

% for the journal paper, we can skip long introduction to variable stars and RR Lyrae
% for "radnja", more text (in Croatian!) will be needed...

RR Lyrae stars are pulsating variable stars with periods in the range 3--30 hours and large amplitudes
that increase towards blue optical bands (e.g., in the SDSS $g$ band from 0.2 mag to 1.5 mag;
\citealt{2010ApJ...708..717S}). In addition to being interesting astrophysical objects that provide
constraints for the stellar pulsation theory and internal structure calculations, RR Lyrae are an important
population for studying  Galactic structure because they are calibrateable standard candles and sufficiently
bright to be detected at large distances;  thus they represent an excellent tracer of low-metallicity outer
halo population. For comprehensive reviews of RR Lyrae stars, we refer the reader to \cite{1995CAS....27.....S}
and \cite{2009Ap&SS.320..261C}.

RR Lyrae stars often exhibit amplitude, period, and phase modulation, or the so-called Blazhko effect (hereafter,
``Blazhko stars''). For examples of well-sampled observed light curves showing the Blazhko effect, see,  e.g., Kepler
data shown in Figures 1 and 2 from \cite{2010MNRAS.409.1585B}. The reported incidence rates for the Blazhko effect
range from 5\% \citep{2007MNRAS.377.1263S} to 60\% \citep{2014A&A...570A.100S}. For a relatively small sample of
151 stars with Kepler data, a claim has been made that essentially every RR Lyrae star exhibits modulated light curve
\citep{2018A&A...614L...4K}. The difference in Blazhko incidence rates for the two largest samples, obtained
by the OGLE-III survey for the Large Magellanic Cloud (LMC, 20\% out of 17,693 stars; \citealt{2009AcA....59....1S})
and the Galactic bulge (30\% out of 11,756 stars; \citealt{2011AcA....61....1S}), indicate possible variation of
the Blazhko incidence rate with underlying stellar population properties. 

The Blazhko effect has been known for long time \citep{1907AN....175..325B}, but its detailed observational
properties and theoretical explanation of its causes remain elusive \cite{2009AIPC.1170..261K}.
Quoting \cite{2016CoKon.105...61K}: ``As of this writing, we do not have a clue why many RR Lyrae
stars vary their amplitudes that leads in some cases nearly ceasing pulsation in the low-amplitude
states. [...] with no physically justified and testable idea/model we miss some basic ingredient not just
in the pulsation models but most likely in the evolutionary models, too.''. 
References to various proposed models for the mysterious Blazhko effect and main
reasons why they fail to explain observations are summarized in \cite{2016CoKon.105...61K}. 

A part of the reason for incomplete observational description of the Blazhko effect is difficulties in discovering a large number 
of Blazhko stars due to temporal baselines that are too short and insufficient number of observations per object
\citep{2016CoKon.105...61K,2022ApJS..258....4H}. With the advent of modern sky surveys, a number of studies
reported large increases in the number of known Blazhko stars, starting with a sample of about 700 Blazhko
stars discovered by the MACHO survey towards LMC \citep{2003ApJ...598..597A} and about 500 Blazhko stars
discovered by the OGLE-II survey towards the Galactic bulge \citep{2003AcA....53..307M}. 
Most recently,  about 4,000 Blazhko stars were discovered in the Large and Small Magellanic Clouds
\citep{2009AcA....59....1S, 2010AcA....60..165S}, and an additional $\sim$3,500 stars were discovered in the
Galactic bulge \citep{2011AcA....61....1S}, both by the OGLE-III survey. Nevertheless, discovering the Blazhko
effect in field RR Lyrae stars that are spread over the entire sky remains a much harder problem: only about
200 Blazhko stars in total from all the studies of field RR Lyrae stars have been reported so far (see Table 1
in \citealt{2016CoKon.105...61K}). 

Here we report the results of a search for the Blazhko effect in a sample of $\sim$2,000 field RR Lyrae stars with
LINEAR and ZTF light curve data. A preliminary subsample of about $\sim$800 stars was selected using various
light curve statistics, and then $\sim$200 stars were confirmed visually as displaying the Blazhko effect. This new
sample doubles the number of known field RR Lyrae stars that exhibit the Blazhko effect. In \S\ref{sec:analysis}
we describe our datasets and analysis methodology, and in \S\ref{sec:results} we present our analysis results. 
Our main results are summarized and discussed in \S\ref{sec:discussion}. 





