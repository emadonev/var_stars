
\section{Discussion and Conclusions\label{sec:discussion}}

The reported incidence rates for the Blazhko effect
range from 5\% \citep{2007MNRAS.377.1263S} to 60\% \citep{2014A&A...570A.100S}. For a relatively small sample of
151 stars with Kepler data, a claim has been made that essentially every RR Lyrae star exhibits modulated light curve
\citep{2018A&A...614L...4K}. The difference in Blazhko incidence rates for the two most extensive samples, obtained
by the OGLE-III survey for the Large Magellanic Cloud (LMC, 20\% out of 17,693 stars; \citealt{2009AcA....59....1S}).
Moreover, the Galactic bulge (30\% out of 11,756 stars; \citealt{2011AcA....61....1S}) indicates a possible variation of
the Blazhko incidence rate with underlying stellar population properties. In this work, 7.75\% of the original RR Lyrae dataset are Blazhko stars. Since our sample size is considerable, we conclude that the incidence rate of Blazhko stars in our work is representative and aligns with other works. We theorize that the difference in incidence rates occurs due to varying data precision, the temporal baseline length, and differences in visual or algorithmic analysis.
We also conclude that our algorithm's success rate in finding 228 out of 531 potential Blazhko stars is 43\% . This high number indicates that the algorithm is quite successful and can be used and refined further for efficient Blazhko star selection. 

For future research, we would like to explore the final finding and find a connection or a factor that might give rise to a mechanism that explains the Blazhko effect. Due to the significant time difference between LINEAR and ZTF observing times ( around 15 years difference ), stars
where the effect is found in both datasets might prove very interesting as they show the Blazhko effect to be long lasting and present for long periods of time in relation to the short period of RR Lyrae stars. 
Based on the final finding, we confirm that the light curve modulation can be unstable, as discussed by \cite{2009MNRAS.400.1006J} 
The project is an excellent example of automatizing the search for Blazhko stars. It can further be improved by training a neural network to replace visual analysis, and our current algorithms can be improved with other models. This work can provide a base for finding more Blazhko stars for the future Vera Rubin observatory. The Legacy Survey of Space and Time (LSST; \citealt{2019ApJ...873..111I}) will be an excellent survey for studying Blazhko effect
\citep{2022ApJS..258....4H} because it will have both a long temporal
baseline (10 years) and a large number of observations per object
(nominally 825; LSST Science Requirements Document\footnote{Available as ls.st/srd}).

A prominent issue in this work is a short temporal baseline which results in few datapoints, limiting our ability to find small changes or modulations in the data.
With time-resolved photometry expected from LSST, a similar analysis will be performed for
RR Lyrae stars in the southern sky and we anticipate a higher fraction of discovered Blazhko stars due to better sampling
and superior photometric quality, since the incidence rate of the Blazhko effect increases with sensitivity to small-amplitude modulation, and thus with
photometric data quality \citep{2009MNRAS.400.1006J}.


\citep{2020MNRAS.494.1237S} classify Blazhko stars in 6 classes using the morphology of their amplitude modulation
(though we note that the most dominant class includes 90\% of the sample). They find bimodal distribution of Blazko periods, with two components
centered on 48 d and 186 d.

 LINEAR and ZTF data used here do not
 have as many data points as OGLE-III used by them (comment earlier, in Introduction? also Kepler is great, \citep{2010MNRAS.409.1585B}).
 

From \cite{2009MNRAS.400.1006J}: A sample of 30 RRab stars was extensively observed, and light-curve modulation was detected in 14 cases. The 47 per cent occurrence rate of the modulation is much larger than any previous estimate. 