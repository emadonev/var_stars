
\section{Discussion and Conclusions\label{sec:discussion}}

We found excellent agreement
between the best-fit periods for RR Lyrae stars estimated separately from LINEAR and ZTF light curves. 
The sample of 228 stars presented here nearly doubles the number of field RR Lyrae stars displaying the Blazhko
effect and  places a lower limit of (11.4$\pm$0.8)\% for their incidence rate. 
The reported incidence rates for the Blazhko effect
range from 5\% \citep{2007MNRAS.377.1263S} to 60\% \citep{2014A&A...570A.100S}.
Differences in reported incidence rates can occur due to varying data precision, the temporal baseline length, and differences in visual or algorithmic analysis.
For a relatively small sample of
151 stars with Kepler data, a claim has been made that essentially every RR Lyrae star exhibits modulated light curve
\citep{2018A&A...614L...4K}. The difference in Blazhko incidence rates for the two most extensive samples, obtained
by the OGLE-III survey for the Large Magellanic Cloud (LMC, 20\% out of 17,693 stars; \citealt{2009AcA....59....1S})
and the Galactic bulge (30\% out of 11,756 stars; \citealt{2011AcA....61....1S}) indicates a possible variation of
the Blazhko incidence rate with underlying stellar population properties. 
 
We find that ab type RR Lyrae which show the Blazhko effect have about 5\% (0.030 day) shorter periods than starting
sample. While not large, the statistical significance of this difference is 7.1$\sigma$. At a similar uncertainty level
($\sim$1\%), we don't detect period difference for c type stars, and don't detect any difference in amplitude distributions.
We also find that for some stars the Blazhko effect is discernible in only one dataset. This finding  strongly suggests that Blazhko effect can
appear and disappear on time scales shorter than about a decade, in agreement with literature 
\citep{2009MNRAS.400.1006J, 2010A&A...520A.108P, 2014ApJS..213...31B}. 


The LINEAR and ZTF datasets analyzed in this work were sufficiently large that we had to rely on algorithmic
pruning of the initial sample. The sample size problem will be even larger for surveys such as the Legacy Survey
of Space and Time (LSST; \citealt{2019ApJ...873..111I}). LSST will be an excellent survey for studying Blazhko effect
\citep{2022ApJS..258....4H} because it will have both a long temporal baseline (10 years) and a large number of
observations per object (nominally 825; LSST Science Requirements Document\footnote{Available as ls.st/srd}).
We anticipate a higher fraction of discovered Blazhko stars with LSST than reported here due to better sampling
and superior photometric quality, since the incidence rate of the Blazhko effect increases with sensitivity to
small-amplitude modulation, and thus with photometric data quality \citep{2009MNRAS.400.1006J}.

The size and quality of LSST sample will motivate further developments of the selection algorithms. 
One obvious improvement will be inspection of neighboring objects to confirm photometric quality,
as well as inspection of images to test implication of an isolated point source (e.g., blended object photometry
can be affected by variable seeing beyond aperture correction valid for isolated point sources). 
Another improvement is forward modeling of the Blazhko modulation, rather than searching for $\chi^2$
outliers \citep{2011MNRAS.417..974B, 2012MNRAS.424..649G}.
For example, \cite{2020MNRAS.494.1237S} classified Blazhko stars in 6 classes using the morphology
of their amplitude modulation (the most dominant class includes 90\% of the sample). They also found bimodal distribution
of Blazko periods, with two components centered on 48 d and 186 d. These results give hope that forward
modeling of the Blazhko effect will improve the selection of such stars.
 
 