\section{Introduction\label{sec:intro}}

RR Lyrae stars are pulsating variable stars with periods in the range of 3--30 hours and large amplitudes
that increase towards blue optical bands (e.g., in the SDSS $g$ band from 0.2 mag to 1.5 mag;
\citealt{2010ApJ...708..717S}). For comprehensive reviews of RR Lyrae stars, we refer the reader to \cite{1995CAS....27.....S} and \cite{2009Ap&SS.320..261C}.

RR Lyrae stars often exhibit amplitude and phase modulation, or the so-called Blazhko effect\footnote{The Blazhko effect was discovered by Lidiya Petrovna Tseraskaya and first reported by Sergey Blazhko.} (hereafter,
``Blazhko stars''). For examples of well-sampled observed light curves showing the Blazhko effect, see,  e.g., Kepler
data shown in Figures 1 and 2 from \cite{2010MNRAS.409.1585B}. The Blazhko effect has been known for a long time \citep{1907AN....175..325B}, but its detailed observational properties and theoretical explanation of its causes remain elusive
\citep{2008JPhCS.118a2060K,2009AIPC.1170..261K,2014IAUS..301..241S}.
Various proposed models for the Blazhko effect, and principal reasons why they fail to explain observations, are summarized in \cite{2016CoKon.105...61K}. 

A part of the reason for the incomplete observational description of the Blazhko effect is difficulties in discovering a large number 
of Blazhko stars due to temporal baselines that are too short and insufficient number of observations per object
\citep{2016CoKon.105...61K,2022ApJS..258....4H}. With the advent of modern sky surveys, several studies
reported large increases in the number of known Blazhko stars, starting with a sample of about 700 Blazhko
stars discovered by the MACHO survey towards the LMC \citep{2003ApJ...598..597A} and about 500 Blazhko stars
discovered by the OGLE-II survey towards the Galactic bulge \citep{2003AcA....53..307M}. 
Most recently,  about 4,000 Blazhko stars were discovered in the LMC and SMC 
\citep{2009AcA....59....1S, 2010AcA....60..165S}, and an additional $\sim$3,500 stars were discovered in the
Galactic bulge \citep{2011AcA....61....1S, 2017MNRAS.466.2602P}, both by the OGLE-III survey. Nevertheless, discovering the Blazhko
effect in field RR Lyrae stars that are spread over the entire sky remains a much harder problem: only about
400 Blazhko stars in total \citep{2013A&A...549A.101S} from all the studies of field RR Lyrae stars have been reported so far (see also Table 1
in \citealt{2016CoKon.105...61K}). 

Here, we report the results of a search for the Blazhko effect in a sample of $\sim$3,000 field RR Lyrae stars with
LINEAR and ZTF light curve data. A preliminary subsample of about $\sim$500 stars was selected using various
light curve statistics, and then 228 stars were confirmed visually as displaying the Blazhko effect. This new
sample doubles the number of field RR Lyrae stars that exhibit the Blazhko effect. In \S\ref{sec:data}
and \S\ref{sec:analysis} we describe our datasets and analysis methodology, and in \S\ref{sec:results} we present our analysis results. 
Our main results are summarized and discussed in \S\ref{sec:discussion}.
