

\section{Results}\label{sec:results}

Starting with a sample of 2857 field RR Lyrae stars with both LINEAR and ZTF data, we found 228 stars exhibiting
convincing Blazhko effect. Out of these 228, 14 were selected via periodogram, and 214 via the scoring algorithm. 

From Fig. ~ \ref{fig:chi_final} we can see that most Blazhko stars selected via periodogram have very low $\chi^2_{dof}$ values for both 
LINEAR and ZTF data. Meanwhile, Blazhko stars selected via the scoring algorithm generally have low $\chi^2_{dof}$ LINEAR scores, with an average of 1.78, and higher $\chi^2_{dof}$ ZTF values,
with an average of 4.09. Most Blazhko stars are part of the 4-point $\chi^2_{dof}$ range (94 stars), with 66 stars in the 5-point range and 54 in the 3-point range. In accordance with this, the
average Blazhko candidate score is 5.90. Based on the light curve type, 78.95 \% of stars are RRab type, and 21.05 \% are RRc type stars. Ratio of RRab to RRc stars is in accordance with other works.

During visual analysis, we noticed that many Blazhko stars exhibited convincing Blazhko effect either in LINEAR or in ZTF data, with few examples of the effect in both datasets. 
Also, the modulation in light curves isn't constant for most stars, rather it varies throughout the observing season. In the following discussion we discuss how to utilize this finding.

